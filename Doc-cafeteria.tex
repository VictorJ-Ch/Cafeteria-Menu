\documentclass[12pt]{article}
\usepackage[utf8]{inputenc}
\usepackage[spanish]{babel}
\usepackage{graphicx}
\usepackage{hyperref}
\hypersetup{
    colorlinks=true,
    linkcolor=cyan,
    filecolor=magenta,      
    urlcolor=blue,
    }

\title{CAFETERIA\\\emph{Menu}}
\author{V�ctor Chavarria, Samuel Gutierrez, Jonathan Zavala, Grecia Morales}
\date{Septiembre, 2023}

\begin{document}

\maketitle

\newpage
\tableofcontents
\newpage

\section {HW}
\par Caratula
\par Requerimiento funcional
\par Requerimiento no funcional (velocidad de respuesta, sistema de desarrollo)
\par Casos de uso:
\par diagrama
\par flujo ideal
\par flujo alterno

\section {Requerimiento funcional}
\begin{itemize}
  \item Guardado de alimentos en canasta.
  \item Guardado de alimentos en las categor�as.
	\item Guardado de los turnos que se dieron.
	\item Dar turnos que no se han dado.
	\item Reinicio de los turnos que se dan despu�s de 24 horas.
\end{itemize}

\section {Requerimiento no funcional}
\begin{itemize}
  \item Sistema de desarrollo en Visual Studio Code.
\end{itemize}

\section {Casos de uso:}
\subsection {Diagrama}
\par lorem

\subsection {Flujo ideal}
\begin{itemize}
  \item Seleccionar una categor�a.
  \item Seleccionar un alimento.
  \item Agregar a la canasta.
	\item Revisar su pedido.
	\item Obtener tu turno.
\end{itemize}

\subsection {Flujo alterno}
\begin{itemize}
  \item Seleccionar una categor�a.
		\suitem Revusar tu canasta.
  \item Seleccionar un alimento.
		\subitem Revisar la informaci�n nutrimental del aliemento.
		\subitem Agregar el alimento a la canasta.
		\subitem Revisar la canasta.
		\subitem Regresar a la selecci�n de categor�as.
  \item Agregar a la canasta.
		\subitem Seguir comprando.
		\subitem Regresar a la selecci�n de categor�as.
		\subitem Revisar la canasta.
	\item Revisar su pedido.
		\subitem Regresar a la selecci�n de categor�as.
		\subitem Regresar a la selecci�n de alimentos.
	\item Obtener tu turno.
\end{itemize}

\end{document}